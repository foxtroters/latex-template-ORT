
%https://en.wikibooks.org/wiki/LaTeX/Internationalization
\usepackage[utf8]{inputenc}
\usepackage[english,spanish]{babel}
\def\spanishdeactivate{<>."}

%https://www.sharelatex.com/learn/Aligning_equations_with_amsmath
\usepackage{amsmath}
\usepackage{amsfonts}
\usepackage{amssymb}

\usepackage{listingsutf8}%codigo con caracteres especiales


%%ORT
%metadata para la ORT
\usepackage[unicode,
            pdftex]{hyperref}
\hypersetup{ 
 pdfauthor={AUTORES},%TODO
 pdftitle={TITULO},%TODO
 pdfsubject={SUBJECT},%TODO
 pdfkeywords={tag1; tag2; tag3}%TODO
}



%%%%%%%%%%%%%%%%%%% http://tex.stackexchange.com/a/103290
%cambio de formato de titulos
\usepackage{titlesec}

%\titleformat{\paragraph}
%  {\normalfont\fontsize{12}{15}\bfseries}{\theparagraph.}{1em}{}

  %\titleformat*{\paragraph}
  %{\normalfont\fontsize{12}\bfseries}{\theparagraph.}{1em}{}
  
  
  \titleformat{\chapter}[display]
{\normalfont\huge\bfseries}{\chaptertitlename\ \thechapter}{20pt}{\Huge}

\titleformat{\section}
{\normalfont\fontsize{20}{0}\bfseries}{\thesection}{1em}{}

\titleformat{\subsection}
{\normalfont\fontsize{18}{0}\bfseries}{\thesubsection}{1em}{}

\titleformat{\subsubsection}
{\normalfont\fontsize{15}{0}\bfseries}{\thesubsubsection}{1em}{}

\titleformat{\paragraph}
{\normalfont\fontsize{13}{0}\bfseries}{}{1em}{}%no se muestra el numero, para agregarlo incluir \theparagraph

\titleformat{\subparagraph}
{\normalfont\fontsize{11}{0}\bfseries}{}{1em}{}%no se muestra el numero, para agregarlo incluir \thesubparagraph

%espacios despues de titulos
%defaults en http://ctan.dcc.uchile.cl/macros/latex/contrib/titlesec/titlesec.pdf dentro de Standard Classes
\titlespacing*{\chapter}        {0pt}{50pt}{40pt}
\titlespacing*{\section}        {0pt}{3.5ex plus 1ex minus .2ex} {2.3ex plus .2ex}
\titlespacing*{\subsection}     {0pt}{3.25ex plus 1ex minus .2ex}{1.5ex plus .2ex}
\titlespacing*{\subsubsection}  {0pt}{3.25ex plus 1ex minus .2ex}{1.5ex plus .2ex}
\titlespacing*{\paragraph}      {0pt}{3.25ex plus 1ex minus .2ex}{1.5ex plus .2ex}
\titlespacing*{\subparagraph}   {0pt}{3.25ex plus 1ex minus .2ex}{1.5ex plus .2ex}

%%%%%%%%%%%%%%%%%%%%%%%%%%%%%%%%%%%%%%%%%


%%posición de la numeración
\usepackage{fancyhdr}
\fancypagestyle{plain}{% Redefining plain page style
  \fancyhf{} %clear all header and footer fields
  \fancyfoot[RO]{\thepage}
}%
\fancyhf{} %clear all header and footer fields
\fancyfoot[RO]{\thepage}
\renewcommand{\headrulewidth}{0pt}
\renewcommand{\footrulewidth}{0pt}
\pagestyle{fancy}

%%Nombre del índice tabla de contenido
\renewcommand\contentsname{Índice}

%%para incluir imagenes
\usepackage{graphicx}

%%para la bibliografía
%%cambia el nombre
\renewcommand\bibname{Referencias Bibliográficas}
%%la agrega al indice
\usepackage[nottoc,numbib]{tocbibind}

%%para los capítulos
\makeatletter
\def\@makechapterhead#1{%
  \vspace*{50\p@}%
  {\parindent \z@ \raggedright \normalfont
    \ifnum \c@secnumdepth >\m@ne
        \huge\bfseries \space \thechapter\space
    %    \par\nobreak
    %    \vskip 20\p@
    \fi
    \interlinepenalty\@M
    \Huge \bfseries #1\par\nobreak
    \vskip 40\p@
  }}
  \makeatother
  
  
  
%%para poner letras no ascii en el código fuente 
\lstset{
  literate=
           {á}{{\'a}}1
           {é}{{\'e}}1
           {í}{{\'i}}1
           {ó}{{\'o}}1
           {ú}{{\'u}}1
           {ñ}{{\~n}}1
           {Á}{{\'A}}1
           {É}{{\'E}}1
           {Í}{{\'I}}1
           {Ó}{{\'O}}1
           {Ú}{{\'U}}1
           {Ñ}{{\~N}}1
}



  
%%%%%%%%%%%%%%%%%%%%%%%%%%%%%%%%%%%%%%%%%%%
 \setcounter{secnumdepth}{5}
 \usepackage[table,xcdraw]{xcolor}

%%%%%%%%%%%%%%%%%% ecuaciones mas grandes
% http://tex.stackexchange.com/a/160440
%uso
%\begin{equationBig}
%        {\frac {1}{2}}
%\end{equationBig}
\usepackage{graphicx}
\usepackage{environ}
\NewEnviron{equationBig}{%
    \begin{equation*}
    \scalebox{1.8}{$\BODY$}
    \end{equation*}
    }
    
    
%%%%%%%%%%%%%%%%%%%%%%Inclusion de codigo
\usepackage{listings}
\usepackage{color}


\definecolor{mygreen}{rgb}{0,0.6,0}
\definecolor{mygray}{rgb}{0.5,0.5,0.5}
\definecolor{mymauve}{rgb}{0.58,0,0.82}
\definecolor{string}{rgb}{0.47,0.43,0.32}

%\lstset{ %
%	backgroundcolor=\color{white},   % choose the background color
%	basicstyle=\ttfamily,        % size of fonts used for the code
%	breaklines=true,                 % automatic line breaking only at whitespace
%	captionpos=b,                    % sets the caption-position to bottom
%	commentstyle=\color{mygreen},    % comment style
%	escapeinside={\%*}{*)},          % if you want to add LaTeX within your code
%	keywordstyle=\color{blue},       % keyword style
%	stringstyle=\color{string},     % string literal style
%	morecomment=[l]//,
%	otherkeywords={ static, finish , async , for , while , here , at , var , val, public, private, protected, def},
%}

\definecolor{lightgray}{rgb}{.9,.9,.9}
\definecolor{darkgray}{rgb}{.4,.4,.4}
\definecolor{purple}{rgb}{0.65, 0.12, 0.82}

\lstdefinelanguage{x10}{
  keywords={static, new, true, false, catch, def, return, null, switch, var, val, if, in, while, for, do, else, case, break, class, throw, this, private, protected, public},
  keywordstyle=\color{blue}\bfseries,
  ndkeywords={finish, async, at, here},
  ndkeywordstyle=\color{mymauve}\bfseries,
  identifierstyle=\color{black},
  sensitive=false,
  comment=[l]{//},
  morecomment=[s]{/*}{*/},
  commentstyle=\color{mygreen}\ttfamily,
  stringstyle=\color{string}\ttfamily,
  morestring=[b]',
  morestring=[b]"
}

\lstset{
   language=x10,
   %backgroundcolor=\color{},
   extendedchars=true,
   basicstyle=\footnotesize\ttfamily,
   showstringspaces=false,
   showspaces=false,
   numbers=left,
   numberstyle=\footnotesize,
   numbersep=9pt,
   tabsize=2,
   breaklines=true,
   showtabs=false,
   captionpos=b,
   columns=flexible,
   keepspaces=true
}

\makeatletter
\def\lst@outputspace{{\ifx\lst@bkgcolor\empty\color{white}\else\lst@bkgcolor\fi\lst@visiblespace}}
\makeatother

%%%%%%%%%%%%%%%%%%%%%%%% graficos
\usepackage{geometry}
\usepackage[utf8]{inputenc}

\usepackage{pgfplots}
\usepackage{pgf-pie}
\usepackage{pgfplotstable}

\pgfplotsset{width=10cm,compat=1.9}

\usepackage{booktabs}
%\usepackage{graphicx}%se incluye antes



%%%%%%%%%%%%%%%%%%%%%%%% cambiar idioma cosas
%\renewcommand{\figurename}{Fig.}
\renewcommand\spanishfigurename{Figura}
\renewcommand\spanishtablename{Tabla}
\addto\captionsspanish{%http://stackoverflow.com/a/9106469/4671917
  \renewcommand{\lstlistingname}%
{Código}%
 }

%%%%%%%%%%%%%%%%%%%%%%%% listas con mas elementos
\usepackage{enumitem}


\setlistdepth{9}

\newlist{noLabel}{enumerate}{9}
\setlist[noLabel,1]{label={}}
\setlist[noLabel,2]{label={}}
\setlist[noLabel,3]{label={}}
\setlist[noLabel,4]{label={}}
\setlist[noLabel,5]{label={}}
\setlist[noLabel,6]{label={}}
\setlist[noLabel,7]{label={}}
\setlist[noLabel,8]{label={}}
\setlist[noLabel,9]{label={}}


%%%%%%%%%%%%%%%%%%%%%%%%%% para que las figuras no se pusheen al final
\usepackage[section]{placeins} %http://tex.stackexchange.com/a/98992
%permite inpedir que queden figuras float pendientes depues de un punto, uso:
%\FloatBarrier



%%%%%%%%%%%%%%%%%%%%%%%%%% para evitar problemas con babel y comillas en los nombres de archivos
\usepackage{grffile} %http://tex.stackexchange.com/a/4132

%%%%%%%%%%%%%%%%%%%%%%%%%% para poner un recuadro al rededor del texto
\usepackage{framed} % http://tex.stackexchange.com/a/53843

%%%%%%%%%%%%%%%%%%%%%%%%%% para poder hacer captions de mas de una linea
\usepackage{caption} % http://tex.stackexchange.com/a/112660

%%%%%%%%%%%%%%%%%%%%%%%%%% para que el espacio entre parrafos se mas grande
%\usepackage[parfill]{parskip}
%\setlength{\parskip}{0.8em}%se usa eso, pero se hace el cambio en el Main despues de la tabla de contenido. %https://www.sharelatex.com/learn/Paragraph_formatting
%\setlength{\parskip}{\baselineskip}

%%%%%%%%%%%%%%%%%%%%%%%%%% para que las figuras queden centradas
%http://www.ctan.org/pkg/floatrow
\usepackage{floatrow}%http://tex.stackexchange.com/a/2652

%%%%%%%%%%%%%%%%%%%%%%%%%% para importar archivos (no usarlo daba problemas en sharelatex)
%https://www.sharelatex.com/learn/Management_in_a_large_project
\usepackage{import}

%%%%%%%%%%%%%%%%%%%%%%%%% para crear tablas complejas
%http://www.tablesgenerator.com/ para crear tablas complejas
\usepackage{multirow}

%%%%%%%%%%%%%%%%%%%%%%%%% para tener ecuaciones de mas de una linea
%http://tex.stackexchange.com/a/3793
\usepackage{amsmath}


%%%%%%%%%%%%%%%%%%%%%%%%% linea de timepo http://tex.stackexchange.com/a/198372
\newcommand\ytl[2]{%con texto del lado izquierdo de la linea
\parbox[b]{8em}{\hfill{\color{black}\bfseries\sffamily #1}~$\cdots\cdots$~}\makebox[0pt][c]{$\bullet$}\vrule\quad \parbox[c]{6cm}{\vspace{7pt}\color{red!40!black!80}\raggedright\sffamily #2\\[7pt]}\\[-3pt]}

\newcommand\ytll[1]{%sin texto del lado izquierdo de la linea
\parbox[b]{8em}{\hfill{}}\makebox[0pt][c]{$\bullet$}\vrule\quad \parbox[c]{6cm}{\vspace{7pt}\color{red!40!black!80}\raggedright\sffamily #1\\[7pt]}\\[-3pt]}


%%%%%%%%%%%%%%%%%%%%%%%% Para incluir pdfs de multiples paginas (de a una pagina a la vez)
\usepackage{pdfpages}

%%%%%%%%%%%%%%%%%%%%%% ancho columnas
%http://tex.stackexchange.com/a/41761
\makeatletter

\newcolumntype{1}{@{\hskip\tabcolsep\vrule width 1pt\hsize=0\tabcolsep}}
\makeatother


%%%%%%%%%%%%%%%%%%%%% Define un nuevo comando, que permite crear subseciones que no son mostradas en la tabla de contenido, ideal para los anexos
%http://stackoverflow.com/a/3805470


\newcommand{\hiddensubsection}[1]{
    \stepcounter{subsection}
    \subsection*{\arabic{chapter}.\arabic{section}.\arabic{subsection}\hspace{1em}{#1}}
}



%%%%%%%%%%%%%%%%%% separacion entre numero y texto en el indice
%http://tex.stackexchange.com/a/16293
\usepackage{tocstyle}
\usetocstyle{standard}


%%%%%%%%%%%%%%%%%% impide que se separe un parrafo en dos paginas si es menor a 3 lineas
%http://tex.stackexchange.com/a/21985
\widowpenalties 4 10000 10000 5000 0
\raggedbottom

%%%%%%%%%%%%%%%%%% permite usar ifthenelse
%http://tex.stackexchange.com/a/58629
\usepackage{xifthen}

%%%%%%%%%%%%%%%%% referencias con nombre y numero
%http://tex.stackexchange.com/a/121871
\newcommand*{\fullref}[2][]{
\ifthenelse{\equal{#1}{}}
{\hyperref[{#2}]{\ref*{#2} \nameref*{#2}}}
{\hyperref[{#2}]{\ref*{#2} {#1}}}
}

%%%%%%%%%%%%%%%% proxima linea en \item y otros casos
\newcommand*{\nextLine}[0]{\mbox{}\\}